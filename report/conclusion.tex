\chapter{Conclusions}\label{C:con}

In conclusion, we have developed a model capable of representing any one-shot game as the result of a series of sequential posturing steps between two or more actors. This dynamic system allows a thorough examination of the role of communication and understanding in decision-making. We have determined that both are necessary precautions in interacting effectively with others; to make the best possible decision available when others are involved, having knowledge of the people you are dealing with, and talking with them before making the decision is critical for both mutual and personal gain. Furthermore, enacting a form of control over another can net an individual with a greedy solution to a problem more beneficial to their interests. However, if both actors have equal power to directly influence one another, a discussion can break down, resulting in no decision being made - cooperative or otherwise.

Looking forward, a number of questions are left to answer around the topic. The exact role of long-term thinking in decision-making is still undefined. It is clear that short-term thinking leads to sub-optimal strategies in complicated scenarios, and that this is exacerbated by the addition of multiple actors. But the limits of this, the potential drawbacks of thinking too far ahead, and potential methods to overcome these limits remain a mystery. Another consideration is that our ability to work with strangers is often influenced by stereotypes based on incomplete knowledge of an individual \cite{burnett2010bootstrapping}. An investigation into the role of incomplete information about opponents in such a model deserves investigation - actors may still be capable of cooperation through assumption, but harmful stereotypes about another may lead to potentially worse results than no knowledge at all.

\chapter{Code Availability}

The code for this project can be found at \url{https://github.com/urn216/StagHunt}. It requires Java 17 to run and may receive updates in the future.
