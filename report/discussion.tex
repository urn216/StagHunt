\chapter{Discussion}\label{C:dsc}

Pregames, through the use of CVI, allow actors to learn the preferences of their counterparts and communicate with them to escape from suboptimal Nash Equilibria in one-shot games. In the Item Swap game, the actors learn to swap in situations with more than two actors when holds are available. Interestingly, though holding actors specifically is not necessary for the two-actor scenario, with more than two, both holding other actors and locking items between actors becomes vital, as without either of these abilities, actors will refuse to swap. This is due to the increased complexity of the multi-actor scenario, where only having access to one's neighbours means there are elements of the environment one actor cannot control. Actors must work together to form chains in order to perform a swap, showing another form of cooperation emerging from these simple swap dances.

The similarities between Item Swaps and Prisoner's Dilemma are little more than superficial. While the exit states are essentially identical between the two games, in PD, actors have no ability to stop another actor from trying to take the greedy strategy. In Item Swapping, actors can simply hold onto their item, removing the ability for their opponent to try for both items. In this way, it is possible to transition from $d,d$ to $c,c$ without traversing directly through $c,d$ or $d,c$. Moreover, reaching the $d,c$ exit condition requires cooperation from both actors. One actor needs to consciously release their item without any holds in place, and the other needs to take it. This is not true for PD, where the difference between $c,c$ and $d,c$ is one move - dependent only on one actor.

Through CVI/SVI, Prisoner's Dilemma actors are able to compromise and agree to cooperate, despite the temptation to defect for a potentially greater reward. The benefits of cooperating are, however, not significant enough to encourage a transition from the defect-defect starting state. Indeed, with greater than two actors, the benefits are seen as less significant still, and actors choose to risk a defect strategy from any starting position. This may be indicative of the same pressures that lead to the 'bystander effect' \cite{fischer2011bystander} - the noted effect where in an emergency, if a large enough crowd is present nobody will call an ambulance. With more people present for an emergency, the pressure of being singled out as the person to call for help may be lessened. This may incentivise people to 'defect' by not spending the energy to engage with the situation, believing somebody else will 'take the fall' and cooperate thus avoiding the poorly valued 'all defect' state.

Looking at the Stag Hunt, we observe a sustainable shift from any starting state to the 'all cooperate' state when CVI/SVI is employed. This is true for all numbers of actors, given a large enough $\gamma$, of which the requirements increase for each subsequent actor in the scenario. This shows that the pregame provides enough of a means of communication to escape from extremely suboptimal Nash Equilibria. In fact, this means of communication is so successful at finding a 'best option' - when one is available - in decision-making that we can confidently say the pregame is an effective way of encouraging actors to cooperate. This heavily implies that communication has a vital role in the process of decision-making in the wider world. People are far more likely to cooperate with one another towards a greater common goal if they can communicate effectively. The chance for miscommunications and incorrect assumptions decreases when actors are able to share knowledge.

We see from our results that pregame posturing has a beneficial outcome for almost every game involving two actors. This can be extrapolated to more actors through the exploration of Item Swapping, Stag Hunting, and the Prisoner's Dilemma as games of interest. It stands to reason that communication therefore plays a vital role in decision-making with more than one person. However, looking at the Item Swappers, we know that on some occasions communication is not enough to encourage an optimal outcome. An affordance in the form of holding - restricting the ability of an actor to leave the pregame - allows the actors to successfully transition to the cooperative optimum.

Interestingly, holds do not always improve the results in the same way communication appears to. In the majority of games, holds are found to be irrelevant to actors with sufficient knowledge of their opponent. In the case of the Prisoner's Dilemma, however, it can be argued that holds reduce the actors' performance within the game. The ability to hold one another results in a never-ending grapple where neither will compromise for anything less than their personal optimum - at the expense of their opponent, who won't let go until the tables are flipped the other way. This results in a stalemate. Even more interestingly, this phenomenon occurs even with the discounting $\gamma$ value below $1.0$ - even when short-term rewards are valued more highly than long-term ones. Beyond a certain point, the actors no longer consider options outside of their starting positions and thus accept whichever result they start with.

Looking back at the role of $\gamma$ for the Stag Hunt problem, we achieve successful cooperation - switching to signalling for hunting stag rather than hares - so long as we allow the value iterator the ability to see far enough into the future for the benefits to outweigh greedy shortsightedness. This is affected directly by the chosen value of $\gamma$ in the VI, as a value of $0.9$ or less results in too great a focus on short-term rewards for actors to agree to cooperate. This is true to an extent for all game representations, where a small $\gamma$ value will limit the ability to cooperate over long pregames. This is important to consider and has ramifications for the role of long-term thinking in cooperation. However, this paper does not delve into such a topic and leaves these ramifications open for analysis.

Another limitation of this study is the assumption of complete knowledge. Between standard value iteration and CVI/SVI, we have two extremes: complete lack of knowledge and complete knowledge. This is seldom the case in reality, and information is gleaned from people as we interact with them. Gaps in this knowledge are often filled with prejudices and stereotypes which can affect our level of trust in a person we otherwise do not know anything about. This topic is discussed in a related field by Burnett et al. in their paper 'Bootstrapping trust evaluations through stereotypes' \cite{burnett2010bootstrapping}. However, a direct analysis with regard to one-shot game theory would still be insightful.
